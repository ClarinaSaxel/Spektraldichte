\begin{frame}
    \frametitle{Motivation}
    \begin{itemize}
        \item Eigenwerte einer Matrix sind in vielen Bereichen der Mathematik und Physik interessant
        \item Oftmals sind Matrizen zu groß um diese effizient zu berechnen
        \item Stattdessen berechnen wir die Spektraldichte (Density of States)
        \item Der Einfachheit halber beschränken wir uns auf symmetrische, reelle Matrizen, die spärlich besetzt sind
    \end{itemize}
\end{frame}

\begin{frame}
    \frametitle{Distribution}
    \begin{definition}[Funktional]
        Sei $V$ ein $\R$-Vektorraum. Ein Funktional $T$ ist eine Abbildung $T: V \to \R$
    \end{definition}
    \begin{definition}[Distribution]
        Sei $\emptyset \neq \Omega \subset \R^n$ offen. Sei $\mathcal{E}$ der Raum der \emph{Testfunktionen} über $\Omega$.\\
        Eine Distribution $T$ ist eine Abbildung $T: \mathcal{E} \to \R$ wobei für alle $g, g_1, g_2, \{g_n\}_{n \in \N} \in \mathcal{E}$
        mit $\lim\limits_{n \to \infty} g_n \to g$ gilt:
        $$T(g_1 + \lambda g_2) = T(g_1) + \lambda T(g_2) \quad \text{und}\quad \lim\limits_{n \to \infty} T(g_n) \to T(g)$$
        Kurz: Eine Distribution $T$ ist ein stetiges und lineares Funktional auf $\mathcal{E}$
    \end{definition}
\end{frame}

\begin{frame}
    \frametitle{Delta-Distribution}
    \begin{definition} [Delta-Distribution]
        Sei $\mathcal{E} = \Cinfty(\Omega)$ mit $0 \in \Omega \subset \R^n$\\
        Dann ist $\delta: \mathcal{E} \to \R, f \mapsto f(0)$ mit $\delta(f) = \langle \delta, f \rangle = f(0)$\\
    \end{definition}
    \vspace{5mm}
    Wichtige Eigenschaft:
    $$\int\limits_{-\infty}^{\infty} f(x) \delta(x-a) \dx = \int\limits_{-\infty}^{\infty} f(x) \delta(a-x) \dx = f(a) \implies \int\limits_{-\infty}^{\infty} \delta(x-a) = 1$$
\end{frame}

% \begin{frame}
%     \frametitle{Trägheitssatz von Sylvester}
%     Sei $V$ ein endlich-dimensionaler Vektorraum über $\C$.\\
%     Sei $s: V \times V \to \C$ eine hermitesche Sesquilinearform\\
%     Sei außerdem $V_0 := \{v \in V: (\forall w \in V) \; s(v,w) = 0\}$ sowie\\
%     $V_{+} := \{v \in V: s(v,v) > 0\}$ und $V_{-} := \{v \in V: s(v,v) < 0\}$\\
%     Dann gilt $$V = V_{+} \oplus V_{-} \oplus V_0$$
%     Hieraus folgt für $A \in \R^{n \times n}$ und $S \in GL(n, \R)$, dass $A$ und $S^TAS$ mit Vielfachheit gezählt die gleichen Anzahlen positiver und negativer Eigenwerte haben.
% \end{frame}

\begin{frame}
    \frametitle{Spektraldichte}
    Sei $A \in \R^{n \times n}$, $A^T = A$ und $A$ spärlich besetzt.\\
    Dann ist die Spektraldichte (engl. Denisty of States (DOS)) definiert als 
    \begin{equation}
        \phi(t) = \frac{1}{n} \sum_{j=1}^{n} \delta(t - \lambda_j)
    \end{equation}
    wobei $\delta$ die Delta-Distribution und $\lambda_j$ die Eigenwerte von A in nicht-absteigender Reihenfolge sind.\\
    Die Anzahl der Eigenwerte in einem Intervall $[a, b]$ kann dann wie folgt ausgedrückt werden:
    \begin{equation} \label{eq:nuab}
        \nu_{[a, b]} = \int\limits_a^b \sum_j \delta(t - \lambda_j) \dt \equiv \int\limits_a^b n \phi(t) \dt
    \end{equation}
\end{frame}

\begin{frame}
    \frametitle{Problemstellung}
    \begin{itemize}
        \item Spektraldichte trivial wenn Eigenwerte von A bekannt
        \item Unpraktisch wenn A sehr groß, da Berechnung teuer
        \item Wir brauchen effiziente Alternativen um $\phi(t)$ abzuschätzen
        \item Allerdings: $\phi(t)$ keine "Funktion" im eigentlichen Sinne
    \end{itemize}
\end{frame}

\begin{frame}
    \frametitle{Idee}
    \begin{itemize}
        \item Sei $I \subseteq \R$ das Interval, dass das Spektrum von $A$ beinhaltet.
        \item Teile nun $I$ in kleinere Teilintervalle $[t_i, t_{i+1}]$
        \item Benutze den Silvestreschen Trägheitssatz um die Eigenwerte in jedem Teilintervall zu zählen.
        \item Berechne den Durchschnittswert von $\phi(t)$ in jedem dieser Intervalle mithilfe von %\eqref{eq:nuab}
        \item Für $(t_{i+1} - t_i) \longrightarrow 0$ nähern sich die Histogramme der Spektraldichte.
        \item Problem: Berechnung der Zerlegung $A - t_i I = LDL^T$ für alle $t_i$ ist zu zeitaufwendig.
        \item Besser: $A$ nur mit Vektoren multiplizieren.
    \end{itemize}
\end{frame}

\begin{frame}
    \frametitle{Annahmen}
    \begin{itemize}
        \item Wir betrachten zwei Methoden zur Annäherung der Spektraldichte
        \item Der Einfachheit halber sei im Folgenden immer $A \in \R^{n \times n}$, $A^T = A$
        \item Die Verallgemeinerung auf hermitesche Matrizen ist im Nachhinein unkompliziert
        \item Zunächst die Kernel-Polynom-Methode (KPM)
        \item Danach das klassische Lanczos-Verfahren zur teilweisen Diagonalisierung von A
        \item Schwierig zu beurteilen welche Methode die beste ist
    \end{itemize}
\end{frame}

\begin{frame}
    \frametitle{Kernel-Polynom-Methode (KPM)}
    \begin{itemize}
        \item Formelle polynomiale Erweiterung der Spektraldichte.
        \item Macht von der Moment Matching Methode gebrauch.
        \item Wir zeigen, wie das Lanczos-Spektrokopieverfahren mit der KPM zusammenhängt
        \item Eine weitere Variante ist die Delta-Gauss-Legendre Methode
    \end{itemize}
\end{frame}