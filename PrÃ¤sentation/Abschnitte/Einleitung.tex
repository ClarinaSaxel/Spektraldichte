\begin{frame}
    \frametitle{Motivation}
    \begin{itemize}
        \pause
        \item Spektraldichte einer Matrix in der Physik interessant
        \pause
        \item DOS zeigt Wahrscheinlichkeit für Eigenwerte nahe $p \in \R$
        \pause
        \item Matrizen oft zu groß um Eigenwerte effizient zu berechnen
        \pause
        \item Bedarf an Methoden, die DOS kostengünstig abzuschätzen
        \pause
        \item Beschränkung auf reelle, symmetrische, spärlich besetzte Matrizen
        \pause
        \item Dieser Vortrag basiert auf dem Paper \textsc{Approximating Spectral Densities of Large Matrices} von Lin Lin, Yousef Saad, and Chao Yang 
    \end{itemize}
\end{frame}

\begin{frame}
    \frametitle{Ein paar Definitionen vorab}
    \pause
    \begin{definition}[Funktional]
        Sei $V$ ein $\R$-Vektorraum. Ein Funktional $T$ ist eine Abbildung $T: V \to \R$
    \end{definition}
    \pause
    \begin{definition}[Distribution]
        Sei $\emptyset \neq \Omega \subset \R^n$ offen. Sei $\mathcal{E}$ der Raum der \emph{Testfunktionen} über $\Omega$.\\
        Eine Distribution $T$ ist eine Abbildung $T: \mathcal{E} \to \R$ wobei für alle $g, g_1, g_2, \{g_n\}_{n \in \N} \in \mathcal{E}$
        mit $\lim\limits_{n \to \infty} g_n \to g$ gilt:
        $$T(g_1 + \lambda g_2) = T(g_1) + \lambda T(g_2) \quad \text{und}\quad \lim\limits_{n \to \infty} T(g_n) \to T(g)$$
        Kurz: Eine Distribution $T$ ist ein stetiges und lineares Funktional auf $\mathcal{E}$
    \end{definition}
\end{frame}

\begin{frame}
    \frametitle{Delta-Distribution}
    \pause
    \begin{definition} [Delta-Distribution]
        Sei $\mathcal{E} = \Cinfty(\Omega)$ mit $0 \in \Omega \subset \R^n$\\
        Dann ist $\delta: \mathcal{E} \to \R, f \mapsto f(0)$ mit $\delta(f) = \langle \delta, f \rangle = f(0)$\\
    \end{definition}
    \pause
    \vspace{5mm}
    Wichtige Eigenschaft:
    $$\int\limits_{-\infty}^{\infty} f(x) \delta(x-a) \dx = \int\limits_{-\infty}^{\infty} f(x) \delta(a-x) \dx = f(a) \implies \int\limits_{-\infty}^{\infty} \delta(x-a) \dx = 1$$
\end{frame}

\begin{frame}
    \frametitle{Spektraldichte}
    \pause
    \begin{definition} [Spektraldichte]
        Sei $A \in \R^{n \times n}$, $A^T = A$ und $A$ spärlich besetzt.\\
        Dann ist die Spektraldichte definiert als 
        $$\phi(t) = \frac{1}{n} \sum_{j=1}^{n} \delta(t - \lambda_j)$$
    wobei $\delta$ die Delta-Distribution und $\lambda_j$ die Eigenwerte von A in nicht-absteigender Reihenfolge sind.\\
    \end{definition}
    \pause
    Die Anzahl der Eigenwerte in einem Intervall $[a, b]$ kann dann wie folgt ausgedrückt werden:
    $$\nu_{[a, b]} = \int\limits_a^b \sum_j \delta(t - \lambda_j) \dt \equiv \int\limits_a^b n \phi(t) \dt$$
\end{frame}

\begin{frame}
    \frametitle{Problemstellung}
    \begin{itemize}
        \pause
        \item Spektraldichte trivial wenn Eigenwerte von A bekannt
        \pause
        \item Unpraktisch wenn A sehr groß, da Berechnung teuer
        \pause
        \item Bedarf effizienter Alternativen um $\phi(t)$ abzuschätzen
        \pause
        \item Allerdings: $\phi(t)$ keine "Funktion" im eigentlichen Sinne
    \end{itemize}
\end{frame}

\begin{frame}
    \frametitle{Intuitive Idee}
    \begin{itemize}
        \pause
        \item Sei $\sigma(A) \subseteq I \subseteq \R$
        \pause
        \item Sei $\{t_i\}_{i = 1}^k \subseteq I$ mit $\bigcup_{i = 1}^k [t_i, t_{i+1}]$
        \pause
        \item Zähle Eigenwerte in jedem Teilintervall
        \pause
        \item Berechne Durchschnittswert von $\phi(t)$ in jedem Intervall mit $\nu_{[a, b]}$
        \pause
        \item Histogramme nähern sich Spektraldichte für $(t_{i+1} - t_i) \longrightarrow 0$
        \pause
        \item Problem: Zerlegung $A - t_i I = LDL^T$ für alle $t_i$ zu zeitaufwendig
        \pause
        \item Besser: $A$ nur mit Vektoren multiplizieren
    \end{itemize}
\end{frame}