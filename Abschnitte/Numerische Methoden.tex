\begin{frame}
    \frametitle{Einführung}
    \begin{itemize}
        \item Zwei Klassen, KPM und Lanczos-Spektrokopieverfahren
        \item Zwei Methoden sind äquivalent zur KPM
        \item Die andere Klasse basiert auf der partiellen Tridiagonalisierung
        \item Zwei Methoden, Gaußscher und Lorentzscher Weichzeichnung (Regularisierung)
        \item Alle Methoden benutzen eine stochastische Sampling-Methode, die auf folgendem Resultat basiert:
    \end{itemize}
\end{frame}

\begin{frame}
    \frametitle{Theorem}
    Sei $A \in \R^{n \times n}$ mit Spektralzerlegung $A = \sum_{j = 1}^n \lambda_j u_j u_j^T$ wobei $u_iu_j^T = \delta_{ij}$.\\
    Sei außerdem $v \in \R^n$ mit $v = \sum_{j=1}^n \beta_j u_j$\\
    Gilt $v_i \sim \mathcal{N}(0, 1)$ i.i.d. für die Komponenten $\{v_i\}_{i = 1}^n$ von $v$, also $\E[v] = 0$ und $\E[vv^T] = \1_n$, dann
    $$\E[\beta_i \beta_j] = \delta_{ij}, \quad i,j \in \N_1^n$$
\end{frame}