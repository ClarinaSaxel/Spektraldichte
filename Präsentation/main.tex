\documentclass{beamer}

% Präsentationseinstellungen
\usetheme{Boadilla}
\usecolortheme{rose}
\usepackage{mathtools}
\usepackage{dsfont}
\everymath{\displaystyle}
\usefonttheme[onlymath]{serif}
\beamertemplatenavigationsymbolsempty

\DeclareMathOperator\arcosh{arcosh}
\DeclareMathOperator\Spur{Spur}

\newcommand{\1}{\mathds{1}}
\newcommand{\dt}{\;\mathrm{d}t}
\newcommand{\dx}{\;\mathrm{d}x}
\newcommand{\Cinfty}{\mathcal{C}^{\infty}}
\newcommand{\im}{\text{Im}}
\newcommand{\C}{\mathbb{C}}
\newcommand{\E}{\mathbb{E}}
\newcommand{\K}{\mathbb{K}}
\newcommand{\N}{\mathbb{N}}
\newcommand{\R}{\mathbb{R}}
\newcommand{\SR}{\mathcal{S}}

\renewcommand{\mid}{\;\middle|\;}

\title{Spektraldichte großer Matrizen}
\subtitle{Eine numerische Annäherung}
\author{Carina Seidel}
\institute[Universität Potsdam]{Universität Potsdam}
\date[3. Juli 2024]{3. Juli 2024}
\logo{\includegraphics[height = 1.5cm]{Uni Potsdam.png}}

\begin{document}

\begin{frame}
\titlepage
\end{frame}
\begin{frame}
\frametitle{Inhaltsverzeichnis}
\tableofcontents
\end{frame}

\section{Einleitung}

\begin{frame}
    \frametitle{Motivation}
    \begin{itemize}
        \item Eigenwerte einer Matrix sind in vielen Bereichen der Mathematik und Physik interessant
        \item Oftmals sind Matrizen zu groß um diese effizient zu berechnen
        \item Stattdessen berechnen wir die Spektraldichte (Density of States)
    \end{itemize}
\end{frame}

\begin{frame}
    \frametitle{Delta Distribution}
    Sei $\mathcal{E} = \Cinfty(\Omega)$ mit $0 \in \Omega \subset \K^n$\\
    Dann ist $\delta: \mathcal{E} \to \K, f \mapsto f(0)$ mit $\delta(f) = \langle \delta, f \rangle = f(0)$\\
    Wichtige Eigenschaft:
    $$\int\limits_{-\infty}^{\infty} f(x) \delta(x-a) \dx = \int\limits_{-\infty}^{\infty} f(x) \delta(a-x) \dx = f(a) \implies \int\limits_{-\infty}^{\infty} \delta(x-a) = 1$$
\end{frame}

% \begin{frame}
%     \frametitle{Trägheitssatz von Sylvester}
%     Sei $V$ ein endlich-dimensionaler Vektorraum über $\C$.\\
%     Sei $s: V \times V \to \C$ eine hermitesche Sesquilinearform\\
%     Sei außerdem $V_0 := \{v \in V: (\forall w \in V) \; s(v,w) = 0\}$ sowie\\
%     $V_{+} := \{v \in V: s(v,v) > 0\}$ und $V_{-} := \{v \in V: s(v,v) < 0\}$\\
%     Dann gilt $$V = V_{+} \oplus V_{-} \oplus V_0$$
%     Hieraus folgt für $A \in \R^{n \times n}$ und $S \in GL(n, \R)$, dass $A$ und $S^TAS$ mit Vielfachheit gezählt die gleichen Anzahlen positiver und negativer Eigenwerte haben.
% \end{frame}

\begin{frame}
    \frametitle{Spektraldichte}
    Sei $A \in \R^{n \times n}$, $A^T = A$ und $A$ spärlich besetzt.\\
    Dann ist die Spektraldichte (engl. Denisty of States (DOS)) definiert als 
    \begin{equation}
        \phi(t) = \frac{1}{n} \sum_{j=1}^{n} \delta(t - \lambda_j)
    \end{equation}
    wobei $\delta$ die Dirac Verteilung und $\lambda_j$ die Eigenwerte von A in nicht-absteigender Reihenfolge sind.\\
    Die Anzahl der Eigenwert in einem Intervall $[a, b]$ kann dann wie folgt ausgedrückt werden:
    \begin{equation} \label{eq:nuab}
        \nu_{[a, b]} = \int\limits_a^b \sum \delta(t - \lambda_j) \dt \equiv \int\limits_a^b n \phi(t) \dt
    \end{equation}
\end{frame}

\begin{frame}
    \frametitle{Problemstellung}
    \begin{itemize}
        \item Spektraldichte trivial wenn Eigenwerte von A bekannt
        \item Unpraktisch wenn A sehr groß, da Berechnung teuer
        \item Wir brauchen effiziente Alternativen um $\phi(t)$ abzuschätzen
        \item Allerdings: $\phi(t)$ keine "Funktion" im eigentlichen Sinne
    \end{itemize}
\end{frame}

\begin{frame}
    \frametitle{Idee}
    \begin{itemize}
        \item Sei $I \subseteq \R$ das Interval, dass das Spektrum von $A$ beinhaltet.
        \item Teile nun $I$ in kleinere Teilintervalle $[t_i, t_{i+1}]$
        \item Benutze den Silvestreschen Trägheitssatz um die Eigenwerte in jedem Teilintervall zu zählen.
        \item Berechne den Durchschnittswert von $\phi(t)$ in jedem dieser Intervalle mithilfe von %\eqref{eq:nuab}
        \item Für $(t_{i+1} - t_i) \longrightarrow 0$ nähern sich die Histogramme der Spektraldichte.
        \item Problem: Berechnung der Zerlegung $A - t_i I = LDL^T$ für alle $t_i$ ist zu zeitaufwendig.
        \item Besser: $A$ nur mit Vektoren multiplizieren.
    \end{itemize}
\end{frame}

\begin{frame}
    \frametitle{Annahmen}
    \begin{itemize}
        \item Wir betrachten zwei Methoden zur Annäherung der Spektraldichte
        \item Der Einfachheit halber sei im Folgenden immer $A \in \R^{n \times n}$, $A^T = A$
        \item Die Verallgemeinerung auf hermitesche Matrizen ist im Nachhinein unkompliziert
        \item Zunächst die Kernel-Polynom-Methode (KPM)
        \item Danach das klassische Lanczos-Verfahren zur teilweisen Diagonalisierung von A
        \item Schwierig zu beurteilen welche Methode die beste ist
    \end{itemize}
\end{frame}

\begin{frame}
    \frametitle{Kernel-Polynom-Methode (KPM)}
    \begin{itemize}
        \item Formelle polynomiale Erweiterung der Spektraldichte.
        \item Macht von der Moment Matching Methode gebrauch.
        \item Wir zeigen, wie das Lanczos-Spektrokopieverfahren mit der KPM zusammenhängt
        \item Eine weitere Variante ist die Delta-Gauss-Legendre Methode
    \end{itemize}
\end{frame}

\section{Die Kernel-Polynom-Methode}

\section{Überblick}
Bei der sogenannten Kernel-Polynom-Methode, kurz KPM, handelt es sich vielmehr um eine Klasse von Methoden, die mehrere Varianten umfasst.
Wir werden im Folgenden die allgemeine Herangehensweise betrachten.\\
Wie der Name bereits nahelegt, ist die KPM eine polynomiale Erweiterung der Spektraldichte.
Dabei werden die Koeffizienten der Polynome aus der Momentenmethode abgeleitet, um wie in der Statistik eine Schätzfunktion zu erhalten.
Diese Methode beruht auf einem Resultat aus dem folgendem Theorem:


\begin{theorem}
    Sei $A = A^T \in \R^{n \times n}$ mit Spektralzerlegung
    $$A = U \Lambda U^T \quad \text{wobei} \quad UU^T = \1_n \text{ und } \Lambda = \diag(\lambda_1, ..., \lambda_n)$$ 
    Seien außerdem $\beta, v \in \R^n$ mit $v = U\beta$.\\
    Gilt $v_i \sim_\text{i.i.d.} \mathcal{N}(0, 1)$ für die Komponenten $\{v_i\}_{i = 1}^n$ von $v$, also
    \begin{equation} \label{eq:normalverteiltervektor}
        \E[v] = 0 \text{ und } \E[vv^T] = \1_n \text{,}
    \end{equation}
    dann
    $$\E[\beta \beta^T] = \1_n$$
\end{theorem}


\begin{proof}[Beweis Theorem 1]
    Es gilt
    $$\E[v] = \E[U\beta] = U\E[\beta] = 0 \implies \E[\beta] = 0$$
    Weiterhin gilt, dass
    $$\E[vv^T] = \E[(U\beta)(U\beta)^T] = \E[U\beta \beta^TU^T] = U \E[\beta \beta^T]U^T = \1_n$$
    Hieraus folgt, dass $\E[\beta \beta^T] = \1_n$
\end{proof}

\vspace{0.5 cm}
Dieses Theorem hat ein schönes Resultat, wenn man nun eine Matrixfunktion $f(A)$ betrachtet.
Dann haben wir

\begin{align*}
    \E\left[v^Tf(A)v\right] = \E\left[(U\beta)^Tf(U\Lambda U^T)(U\beta)\right] & = \E\left[\beta^TU^TUf(\Lambda)U^TU\beta\right]\\
        & = \E\left[\beta^Tf(\Lambda)\beta\right]\\
        & = \E\left[\sum_{j = 1}^n \beta_j^2 f(\lambda_j) \right]\\
        & = \sum_{j = 1}^n f(\lambda_j) \E\left[ \beta_j^2 \right]\\
        & = \sum_{j = 1}^n f(\lambda_j)
\end{align*}

also zusammengefasst
\begin{equation} \label{eq:theoremresultat}
    \E\left[v^Tf(A)v\right] = \Spur(f(A))
\end{equation}

\section{Polynomiale Erweiterung durch Tschebyschev-Polynome}
Aufgrund ihrer vielen einzigartigen Eigenschaften sind Tschebyschev-Polynome besonders gut zur polynomialen Erweiterung der Delta-Distribution geeignet.
Mit Hilfe der trigonometrischen Funktionen können sie auch wie folgt ausgedrückt werden:
\[ T_k(t) =
\begin{cases}
    \cos(k \arccos(t))            & \quad \text{für } k \in [-1, 1]\\
    \cosh(k \arcosh(t))           & \quad \text{für } k > 1\\
    (-1)^k \cosh(k \arcosh(-t))   & \quad \text{für } k < -1
\end{cases}
\]

Wir benutzen im Folgenden nur die Formel $T_k(t) = \cos(k \arccos(t))$.
Daher müssen wir uns auf Matrizen beschränken, deren Eigenwerte im Intervall $[-1, 1]$ liegen.
Sollte diese Voraussetzung nicht erfüllt sein, kann man die Eigenwerte entsprechend transformieren.
Seien dazu $\lambda_{us}$ und $\lambda_{os}$ jeweils die untere bzw. obere Schranke für die Eigenwerte von $A$.
Definiere
$$c := \frac{\lambda_{us} + \lambda_{os}}{2} \quad \text{und} \quad d := \frac{\lambda_{os} - \lambda_{us}}{2}$$
Dann ist $B = \frac{A - c*\1_n}{d}$ eine Matrix mit Eigenwerten im Intervall $[-1, 1]$.
Eine Veranschaulichung dazu ist im Anhang verlinkt.

Tschebyschev-Polynome können zudem mit der Rekursionsformel
$$T_{k + 1}(t) = 2tT_k(t) - T_{k - 1}(t)$$
berechnet werden, wobei die Startbedingunen $T_0(t) = 1$ und $T_1(t) = x$ gelten.

Beachte auch, dass das Resultat in Gleichung \ref{eq:theoremresultat} besagt, dass
\begin{equation} \label{eq:Tschebyschev-Spur}
    \E\left[v^TT_k(A)v\right] = \sum_{j = 1}^n T_k(\lambda_j) = \Spur(T_k(A))
\end{equation}
gilt. Dies ist zentral im weiteren Vorgehen.\\

Sei nun
\begin{equation} \label{eq:Gewichtsfunktion}
    h(x) = \frac{1}{\sqrt{1 - t^2}}
\end{equation}
eine Gewichtsfunktion.
Eine weitere Eigenschaft der Tschebyschev-Polynome ist, dass sie \emph{orthogonal} bezüglich des mit $h$ gewichteten Skalarproduktes

$$\left \langle f, g \right \rangle = \int_{-1}^1 \frac{1}{\sqrt{1 - x^2}} \cdot f(x) \cdot g(x) \dx$$

sind. Das bedeutet, dass

$$\int_{-1}^1 \frac{1}{\sqrt{1 - t^2}} \cdot T_k(t) \cdot T_l(t) \dt =
\begin{cases}
    0               & \quad \text{für } k \neq l\\
    \pi             & \quad \text{für } k = l = 0\\
    \frac{\pi}{2}   & \quad \text{für } k = l \neq 0
\end{cases}$$

\section{Annäherung der Spektraldichte}
Multipliziere nun die Spektraldichte mit dem Inversen der Gewichtsfunktion \ref{eq:Gewichtsfunktion}:
$$\hat{\phi}(t) = \sqrt{1 - t^2} \phi(t) = \sqrt{1 - t^2} \times \frac{1}{n} \sum_{j = 1}^n \delta(t - \lambda_j)$$
Sei nun $g \in \SR$, dem in Definition \ref{def:Schwartz-Raum} beschriebenen Schwartz-Raum,
und $\mu_k \in \R$ Koeffizienten,
die wir nachher berechnen, sodass die folgende Gleichung gilt:
\begin{equation} \label{eq:Distributionsgleichheit}
    \int \limits_{-1}^1 \hat{\phi}(t) g(t) \dt = \int \limits_{-1}^1 \sum_{k = 0}^{\infty} \mu_k T_k(t) g(t) \dt
\end{equation}
Gilt dies für beliebige $g \in \SR$, so vereinfachen wir Gleichung \ref{eq:Distributionsgleichheit} zu
\begin{equation} \label{eq:Tschebyschev-Erweiterung}
    \hat{\phi}(t) = \sum_{k = 0}^{\infty} \mu_k T_k(t)
\end{equation}
Nutze nun die Orthogonalität der Tschebyschev-Polynome aus, um einen bestimmten Koeffizienten $\mu_k$ zu berechnen:
\begin{align*}
    \sum_{l = 0}^{\infty} \mu_l T_l(t) = \hat{\phi}(t) & \implies \left(\sum_{l = 0}^{\infty} \mu_l T_l(t)\right) \cdot T_k(t) = \hat{\phi}(t) \cdot T_k(t)\\
    & \implies \int_{-1}^1 \frac{1}{\sqrt{1 - t^2}} \cdot \left(\sum_{l = 0}^{\infty} \mu_l T_l(t)\right) \cdot T_k(t) \dt = \int_{-1}^1 \frac{1}{\sqrt{1 - t^2}} \cdot \hat{\phi}(t) \cdot T_k(t) \dt\\
    & \implies \mu_k \cdot \frac{\pi}{2 - \delta_{k0}} = \int_{-1}^1 \frac{1}{\sqrt{1 - t^2}} \cdot \sqrt{1 - t^2} \cdot \phi(t) \cdot T_k(t) \dt\\
    & \implies \mu_k = \frac{2 - \delta_{k0}}{\pi} \cdot \int_{-1}^1 \phi(t) \cdot T_k(t) \dt\\
\end{align*}

Durch Anwendung der Delta-Funktion erhält man:
\begin{align*}
    \mu_k = \frac{2 - \delta_{k0}}{\pi} \cdot \int_{-1}^1 \phi(t) \cdot T_k(t) \dt &= \frac{2 - \delta_{k0}}{\pi} \cdot \int_{-1}^1 \frac{1}{n} \sum_{j = 1}^n \delta(t - \lambda_j) \cdot T_k(t) \dt \\
    &= \frac{2 - \delta_{k0}}{n \pi} \sum_{j = 1}^n T_k(\lambda_j)\\
    &= \frac{2 - \delta_{k0}}{n \pi} \Spur(T_k(A))
\end{align*}

Sei nun $n_{vec} \in \R$ und $v_0^{(1)}, v_0^{(2)}, \dots, v_0^{(n_{vec})}$ Vektoren, die die Bedingungen aus dem Theorem erfüllen,
also $\E[v_0^{(k)}] = 0$ und $\E\left[v_0^{(k)}\left(v_0^{(k)}\right)^T\right] = \1_n$.
Aus Gleichung \ref{eq:Tschebyschev-Spur} folgt, dass
$$\zeta_k = \frac{1}{n_{vec}} \sum_{l = 1}^{n_{vec}} \left( v_0^{(l)} \right)^T T_k(A) v_0^{(l)}$$
ein guter Schätzer für $\Spur(T_k(A))$ ist und damit
$$\mu_k \approx \frac{2 - \delta_{k0}}{n \pi} \zeta_k$$

Um die $\zeta_k$ zu bestimmen, sei im Folgenden $v_0 \equiv v_0^{(l)}$
Berechne nun mit Hilfe der Rekursionsformel für Tschebyschev-Polynome:
$$T_{k + 1}(A)v_0 = 2 A T_k(A) v_0 - T_{k - 1}(A) v_0$$
Für $v_k \equiv T_k(A)v_0$ gilt also, dass
$$v_{k + 1} = 2 A v_k - v_{k - 1}$$

Damit sind alle Bauteile zur Berechnung festgelegt und das Ziel der KPM erreicht:
Anstatt rechenaufwendig Matrizen mit anderen Matrizen zu multiplizieren, müssen wir sie nur noch mit Vektoren multiplizieren.
Nun können wir $\phi(t)$ beliebig nah annähren.
Wie bereits erwähnt, ist eine unendlich genaue Annäherung nicht immer wünschenswert.
Wegengilt
$$\lim \limits_{k \to \infty} \mu_k \to 0$$
und wir interessieren uns nur für $T_k(t)$ mit $k \leq M$\\
Daher schätzen wir $\phi$ durch
\begin{equation} \label{eq:Angenäherte Spektraldichte}
    \tilde{\phi}_M(t) = \frac{1}{\sqrt{1 - t^2}} \sum_{k = 0}^{M} \mu_k T_k(t)
\end{equation}
\newline
Der folgende Pseudocode basiert auf \cite[p.~10]{linsaadyang14} und fasst die oben beschriebenen Schritte zusammen.
Ich habe ihn selbst implementiert und im Anhang verlinkt.

\begin{algorithm}
    \caption{Die Kernel-Polynom-Methode}\label{alg:cap}
    \begin{algorithmic}[5]
    \Require $A = A^T \in \R^{n \times n}$ mit Eigenwerten aus dem Intervall $[-1, 1]$
    \Ensure Geschätzte Spektraldichte \{$\tilde{\phi}_M(t_i)$\}\\
    \For{$k = 0 : M$}
    \State $\zeta_k \gets 0$
    \EndFor
    \For{$l = 1 : n_{\text{vec}}$}
    \State $\text{Wähle einen neuen zufälligen Vektor } v_0^{(l)}\text{;}$ \Comment{$v_{0_i}^{(l)} \sim_\text{ i.i.d. } \mathcal{N}(0, 1)$}
    \For{$k = 0 : M$}
    \State $\text{Berechne } \zeta_k \gets \zeta_k + \left( v_0^{(l)} \right)^T v_k{(l)}\text{;}$  
    \If{$k = 0$}
    \State $v_1^{(l)} \gets A v_0^{(l)}$
    \Else
    \State $v_{k+1}^{(l)} \gets 2 A v_k^{(l)} - v_{k-1}^{(l)}$ \Comment{Drei-Term-Rekursion}
    \EndIf
    \EndFor
    \EndFor
    \For{$k = 0 : M$}
    \State $\zeta_k \gets \frac{\zeta_k}{n_{\text{vec}}}$
    \State $\mu_k \gets \frac{2 - \delta_{k0}}{n \pi} \zeta_k$
    \EndFor
    \State $\text{Werte } \tilde{\phi}_M(t_i) \text{ mit Gleichung } \ref{eq:Angenäherte Spektraldichte} \text{ aus} $
    \end{algorithmic}
\end{algorithm}

\section{Qualitätsanalyse der Annäherungen}

\begin{frame}
    \frametitle{Problemstellung}
    \begin{itemize}
        \item Sei im Folgenden $\tilde{\phi}(t)$ eine reguläre Funktion die die Spektraldichte schätzt
        \item Alle Annäherungen sind stetige Funktionen
        \item $\phi(t)$ ist keine Funktion im eigentlichen Sinne
        \item Wir können nicht die $L^p$-Norm benutzen, um $\phi(t) - \tilde{\phi}(t)$ abzuschätzen
        \item Zwei Möglichkeiten, dies zu umgehen
    \end{itemize}
\end{frame}

\begin{frame}
    \frametitle{Schwartz-Raum über $\R$}
    $$\SR(\R) := \left\{f \in \Cinfty(\R) \mid \forall p, k \in \N_0: \sup_{x \in \R} \left| x^pf^{(k)}(x)\right| < \infty \right\}$$
\end{frame}

\begin{frame}
    \frametitle{Erste Methode}
    Wir benutzen die Tatsache, dass $\delta(t)$ eine Verteilung ist:\\
    Sei $g \in \Cinfty(\R)$ eine Testfunktion aus dem Schwartz-Raum $\SR$, dann
    $$\langle \delta(\cdot - \lambda), g \rangle \equiv \int\limits_{-\infty}^{\infty} \delta(t - \lambda) g(t) \dt \equiv g(\lambda)$$
    und für alle $p, k \in \N_0$ $$\sup_{t \in \R} |t^pg^{(k)}(t)| < \infty$$
    Dann wird der Fehler wie folgt gemessen: $$\epsilon_1 = \sup_{g \in \SR} \left| \langle \phi, g \rangle - \langle \tilde{\phi}, g \rangle \right|$$
\end{frame}

\begin{frame}
    \frametitle{Zweite Methode}
    Wir "regularisieren" die $\delta(t)$-Funktionen\\
    Dazu ersetzen wir sie durch stetige und glatte Funktionen\\
    Zum Beispiel die Gaussche Normalverteilung mit einer angemessene Standardabweichung $\sigma$\\
    Die daraus entstandene Funktion $\phi_{\sigma}(t)$ ist wohldefiniert\\
    Für $p=1, 2$ und $\infty$ können wir folgenden Fehler berechnen:
    \begin{equation} \label{eq:eps2}
        \epsilon_2 = \left|\left| \phi_{\sigma}(t) - \tilde{\phi}(t) \right|\right|_p
    \end{equation}
    Diese beiden Methoden sind eng verwandt!
\end{frame}

\begin{frame}
    \frametitle{Der Begriff der Auflösung}
    Selten ist eine exakte Annäherung aller Eigenwerte von $A$ gewünscht.\\
    Oftmals genügt es, die Anzahl der Eigenwerte in einem beliebigen Teilintervall $[a, b]$ des Spektrums zu wissen.\\
    Die Größe $b - a$ dieses Teilintervalls bezeichnet man als Auflösung der Schätzung:\\
    Je kleiner das Teilintervall, desto höher die Auflösung.\\
    Die Genauigkeit der Annäherung ist nur bis zur gewünschten Auflösung aussagekräftig.\\
    Für $\epsilon_2 = \left|\left| \phi_{\sigma}(t) - \tilde{\phi}(t) \right|\right|_p$ aus \eqref{eq:eps2} gilt:\\
    Je kleiner das $\sigma$, desto höher die Auflösung.
\end{frame}

\begin{frame}
    \frametitle{Noch mehr Probleme mit Dirac}
    Betrachte
    $$\nu_{[a, b]} = \int\limits_a^b n \phi(t) \dt$$
    aus \eqref{eq:nuab}. Definiere entsprechend
    $$\tilde{\nu}_{[a, b]} = \int\limits_a^b n \tilde{\phi(t)} \dt$$
    mit $\tilde{\phi}(t) \in \Cinfty$
\end{frame}

\begin{frame}
    \frametitle{Noch mehr Probleme mit Dirac (2)}
    Angenommen, $n = 1$ und $\phi(t) = \delta(t)$.\\
    Unendliche Auflösung bedeutet $\left| \nu_{[a, b]} - \tilde{\nu}_{[a, b]} \right|$ soll für $[a ,b]$ beliebig klein ebenfalls klein sein.\\
    Sei also $a = -\varepsilon, b = \varepsilon$. Aus der Definition der $\delta$-Funktion folgt dann, dass
    $$\lim \limits_{\varepsilon \to 0+} \nu_{[-\varepsilon, \varepsilon]} = 1$$
    während für glatte Funktionen $\tilde{\phi}$ selbstverständlich gilt, dass
    $$\lim \limits_{\varepsilon \to 0+} \tilde{\nu}_{[-\varepsilon, \varepsilon]} = 0$$
    Fazit: Keine glatte Funktion konvergiert zur Spektraldichte unter stetiger Erhöhung der Auflösung
\end{frame}

\begin{frame}
    \frametitle{Einschränkung des Schwartz-Raums}
    Eine endliche Auflösung ist oftmals genug.\\
    Wir können den Schwartz-Raum $\SR$ also einschränken.\\
    Beispiel: Betrachte nur Gaussche Verteilungsfunktionen der Form
    $$g_{\sigma}(t) = \frac{1}{(2\pi\sigma^2)^\frac{1}{2}}e^{-\frac{t^2}{2\sigma^2}}$$
    und schränke $\SR$ auf den Unterraum
    $$\SR(\sigma;[\lambda_{lb}, \lambda_{ub}]) = \left\{ g \mid g(t) \equiv g_{\sigma}(t - \lambda), \lambda \in [\lambda_{lb}, \lambda_{ub}] \right\}$$
    Hierbei sind $\lambda_{lb}$ und $\lambda_{ub}$ jeweils das Infimum und Supremum der Eigenwerte von $A$ und der Parameter $\sigma$ die \emph{Zielauflösung}.\\
    Wir können nun die folgende Metrik zur Qualitätsbewertung nutzen:
    \begin{equation} \label{eq:error}
        E\left[\tilde{\phi};\SR\left(\sigma; \left[\lambda_{lb}, \lambda_{ub} \right] \right)\right] = \sup_{g \in \SR(\sigma;[\lambda_{lb}, \lambda_{ub}])} \left| \langle \phi, g \rangle - \langle \tilde{\phi}, g \rangle \right|
    \end{equation}
\end{frame}

\begin{frame}
    \frametitle{Regularisierung der Spektraldichte}
    \begin{itemize}
        \item Konstruiere zunächst eine glatte Darstellung der $\delta$-Funktion.
        \item Dies muss verhältnismäßig zur gewünschten Auflösung sein.
        \item Der Fehler kann dann direkt berechnet werden
        \item Wahl des $\sigma$: so groß wie möglich für leichte Annäherung, so klein wie möglich für Genauigkeit
    \end{itemize}
\end{frame}

\begin{frame}
    \frametitle{Regularisierung der Spektraldichte mit Gauss}
    Sei $$\phi_{\sigma}(t) = \left \langle \phi(\cdot), g_{\sigma}(\cdot - t)\right \rangle = \sum_{j = 1}^n g_{\sigma}(t - \lambda_j)$$
    Dies ist dann nicht anderes als die "Weichzeichnung" der Spektraldichte durch Gauß-Funktionen der Breite $\sigma$\\
    Genauso sei $$\tilde{\phi}_{\sigma}(t) = \langle \tilde{\phi}(\cdot), g_{\sigma}(\cdot - t) \rangle$$
    Dann ist
    $$E\left[\tilde{\phi};\SR\left(\sigma; \left[\lambda_{lb}, \lambda_{ub} \right] \right)\right] = \sup_{g \in \SR(\sigma;[\lambda_{lb}, \lambda_{ub}])} \left| \langle \phi(\cdot), g_{\sigma}(\cdot - t) \rangle - \langle \tilde{\phi}(\cdot), g_{\sigma}(\cdot - t) \rangle \right|$$
    der $L^\infty$-Fehler zwischen zwei wohldefinierten Funktionen
\end{frame}

\begin{frame}
    \frametitle{Schöne Bilder}
    \includegraphics[height=7.5cm]{screenshot.png}
\end{frame}

\begin{frame}
    \frametitle{Regularisierung der Spektraldichte mit Lorentz}
    Die Lorentz-Funktion ist definiert durch
    $$\frac{\eta}{(t - \lambda)^2 + \eta^2} = -\im \left( \frac{1}{t - \lambda + i \eta} \right) \; ,$$
    wobei $\eta$ eine kleine Regularisierungskonstante ist.\\
    Für $\eta \to 0$ nähert sich die Lorentz-Funktion der Dirac-Funktion um den Eigenwert $\lambda$ an\\
    Dies ist später für die Haydock-Methode relevant.
\end{frame}

\begin{frame}
    \frametitle{Die Bedingung der Nicht-Negativität}
    \begin{itemize}
        \item Die Spektraldichte ist als Wahrscheinlichkeitsverteilung nicht-negativ, also
        $$\forall g \in \SR, g \geq 0: \langle \phi, g \rangle \geq 0$$
        \item Einige numerische Annäherungen brechen mit dieser Eigenschaft
        \item Das führt zu großen Fehlern
    \end{itemize}
\end{frame}

\section{Auswertung}

\end{document}